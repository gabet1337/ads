\documentclass[a4paper,oneside,article,11pt]{memoir}
\usepackage[english]{babel}
\usepackage[utf8]{inputenc}
\usepackage{amsmath,amssymb,amsthm}

% This font looks so good.
\usepackage[sc]{mathpazo}

% Typesetting pseudo-code
\usepackage{algorithm}
\usepackage{algorithmic}
\usepackage{multirow}
% Code comments like [CLRS]
\renewcommand{\algorithmiccomment}[1]{\makebox[5cm][l]{$\triangleright$ \textit{#1}}}
\usepackage{framed,graphicx,xcolor}
\usepackage[font={small,it}]{caption}
\usepackage{listings}
\usepackage{units}
\usepackage{subcaption}
% Relative references
\usepackage{varioref}

\usepackage{hyperref}

\bibliographystyle{alpha}

\title{Advanced Data Structures \\ Project 3 - Theory project 5}
\author{Peter Gabrielsen 20114179}
\newcounter{qcounter}
\begin{document}

\begin{titlingpage}
\clearpage

\maketitle
\thispagestyle{empty}

\begin{abstract}
This project presents solutions to four theoretical exercises. The first exercise asks us to describe how to preprocess a two dimensional array into a data structure that given a rectangular query range can report the minimum in the range, i.e. the 2D range minimum problem. We present a solution in Section~\ref{chp:rmq2d} answering queries in $\mathcal{O}(\sqrt{n})$, where $n$ is the max of number of rows and columns, and uses linear space.

The second problem asks us to describe a data structure which solves the dynamic nearest common ancestor problem under insertion and deletion of leafs. We solve this problem in Section~\ref{chp:LCA} using Link/Cut trees by Sleator and Tarjan\cite{Sleator1985} and are able to do updates and queries in $\mathcal{O}(\log n)$.

In Section~\ref{chp:inplace} we solve the problem of merging two sorted arrays of size $n$ and $m$ respectively in-place. We present a solution to do this in $\mathcal{O}(n+m)$ time.

Finally, in Section~\ref{chp:prefix}, we describe a data structure that supports updating an entry in an array of size $n$ to either $0$ or $1$ such that we can answer queries for the prefix sum modulo $2$, i.e. decide if the sum of a prefix of the array is odd or even, in $\mathcal{O}(\nicefrac{\log n}{\log\log n})$.
\end{abstract}
\end{titlingpage}

\pagebreak

\tableofcontents

\pagebreak

\chapter{Range minimum queries in two dimensions}
\label{chp:rmq2d}
\subsection{Problem description}
\textit{Let $A$ be a two dimensional array of size $n \times m$. Describe how to preprocess $A$ into a data structure that given a rectangular query range given by $(i_1,i_2,j_1,j_2)$ reports the minimum element in $A\left[i_1\dots i_2\right]\left[j_1\dots j_2\right]$. State the space, preprocessing time, and query time of your data structure.}

%\textit{Optional: What is the best space bound you can achieve with $\mathcal{O}(1)$ query time? What is the best query time you can achieve with $\mathcal{O}(nm)$ space?}

\subsection{Solution}
In the rest of this section we let $N = n\cdot m$.

There are several naïve solutions that comes to mind immediately. One such solution is to preprocess all possible ranges using dynamic programming giving $\mathcal{O}(N^2)$ preprocessing and space but with $\mathcal{O}(1)$ query time.
This solution is not reasonable for larger inputs and we must come up with a better solution. Another trivial thing we could do is to extend the idea of using Cartesian Trees\cite{vuillemin80,tarjan84} to solve 1D-RMQ on each row giving $\mathcal{O}(N)$ space and preprocessing time, and $\mathcal{O}(n)$ query time. Demaine~et~al~\cite{demaine09} show that the equivalent of the 1D Cartesian tree does not exist in 2D, and as such we cannot use Cartesian trees in 2D to solve this problem optimally.

Let us instead try to block the two dimensional array into blocks of size $\sqrt{n} \times \sqrt{m}$. This will construct $\Theta(N)$ blocks. For each block we create an optimal solution to the 1D RMQ $\left\langle \mathcal{O}(\sqrt{nm}),\ \mathcal{O}(1)\right\rangle$. On top of the blocks we build a Cartesian tree with leafs being the minimum in each block.
We will now be able to answer queries. A typical query will look as in Figure~\ref{fig:rmq_query}.

\begin{figure}[h!]
\includegraphics[width=\textwidth]{../figures/RMQ_query.png}
\caption{\label{fig:rmq_query}A typical 2D range query. Green is the corners, blue is the sides, and purple is the full blocks.}
\end{figure}

It consists of four corners, four sides, and an area consisting of several full blocks. We will answer the query for each of the nine parts separately and take the minimum of them.
A corner query will typically look as in Figure~\ref{fig:rmq_corner}. This type of query can trivially be answered in $\mathcal{O}(\max(\sqrt{n},\sqrt{m}))$ by taking the minimum of up to $\max(\sqrt{n},\sqrt{m})$ range queries each taking constant time.
\begin{figure}[htbp!]
    \centering
    \begin{subfigure}[b]{0.49\textwidth}
        \includegraphics[width=\textwidth]{../figures/RMQ_corner.png}
        \caption{A typical corner query for 2D RMQ}
        \label{fig:rmq_corner}
    \end{subfigure}
    \begin{subfigure}[b]{0.49\textwidth}
        \includegraphics[width=\textwidth]{../figures/RMQ_side.png}
        \caption{A side in the 2D RMQ query}
        \label{fig:rmq_side}
    \end{subfigure}
    \caption{}
\end{figure}

A side query may look as in Figure~\ref{fig:rmq_side}. The side query may be answered by a single range query to each block on the side since the elements queried is next to each other in the 1D RMQ structure. We know that we have at most $\Theta(\max(\sqrt{m},\sqrt{n}))$ of such queries to make which gives a total time of $\mathcal{O}(\max(\sqrt{m},\sqrt{n}))$ for each side.
Now we only need to handle the full blocks. Since we have built a Cartesian tree on top of the blocks we will be able to answer such queries in a similar way to the other types. We simply make up to $\Theta(\max(\sqrt{m},\sqrt{n}))$ range queries and take the minimum of those.

This solution will require $\mathcal{O}(N)$ space since we use $\sqrt{nm} \cdot \sqrt{nm} = N$ for the $N$ blocks and $N$ space for the Cartesian tree on top of the blocks. We will be able to answer a query in $\mathcal{O}(\max(\sqrt{m},\sqrt{n}))$.

Dividing in blocks of size $\log n \times \log m$ will give a solution with $\left\langle \mathcal{O}(N\log N),\ \mathcal{O}(\log N)\right\rangle$ space and time.

It is shown by Brodal et al\cite{algorithmica12min} that the 2D RMQ can be solved with $\mathcal{O}(N)$ space and $\mathcal{O}(1)$ query time with an additional $\mathcal{O}(N)$ bits.


\chapter{Nearest common ancestors under leaf insertions and deletions}
\label{chp:LCA}
\subsection{Problem description}
\textit{Describe a data structure to maintain a tree $T$ under the insertion and deletion of leafs (to insert a new leaf we are given a pointer to the parent node), while supporting nearest common ancestor queries of two arbitrary nodes in $T$. State the update and query time of your data structure.}

\textit{What are the best update and query bounds you can achieve?}

\subsection{Solution}
Without any preprocessing or smart data structures we can trivially solve this problem in $\mathcal{O}(h)$ where $h$ is the height of the tree by following paths from the nodes to the root until the paths meet. In this case we have constant time insertion and deletion of leafs given a pointer to the parent, but in the worst case the height of the tree is proportional to the size of the tree. On the other end of the scale we can use the reduction from 1D RMQ to nearest common ancestor to answer queries in constant time but insertion and deletion will require linear time in the size of the tree to rebuild the data structure.

After several hours of trying to come up with a better solution and researching solutions it seems like the only solution which allows insertion and deletion is one that involves Link/Cut trees with Splay trees as an auxiliary data structure by Sleator and Tarjan~\cite{Sleator1985}. This solution is sketched in the following section.

\subsection{An $\mathcal{O}(\log n)$ solution}
The link/cut tree represents a forest of trees under the following operations:
\begin{itemize}
	\item{\textit{link(p,q)} - $p$ is a tree root, and $q$ is any node in another tree. This operation makes $p$ a child of $q$.}
	\item{\textit{cut(p)} - $p$ is not a tree root. Delete the edge from $p$ to its parent effectively separating the tree in two.}
	\item{\textit{path(p)} - Do some calculation on the path from $p$ to the root of $p$'s tree such as summing etc.}
	\item{\textit{findroot(p)} - returns the root of the tree containing $p$.}
\end{itemize}

These operations are all supported in $\mathcal{O}(\log n)$. We can use link and cut to support insertion by adding a tree containing a single node to the forest and then linking the trees at the parent pointer given, and deletion by cutting a leaf off and deleting the tree containing the single node.

Using the terminology of the paper the "real tree" is the tree that the above operations operate on, i.e. the forest of trees. In memory the structure is represented in quite a different way. This representation is called the "virtual tree". The virtual tree has the same nodes as the real tree, but they are linked in a different way. It has solid and dashed edges. The connected components of solid edges are represented as Splay trees. The root of each Splay tree is connected via a dashed edge to the root of the connected component on the path to the root. An illustration of a Link/Cut tree with dashed and solid edges can be found in Figure~\ref{fig:lca_linkcut}.

\begin{figure}
\includegraphics[width=\textwidth]{../figures/linkcut-tree.png}
\caption{\label{fig:lca_linkcut}A Link/Cut tree with solid edges between connected components each represented by a Splay tree, and dashed edges connecting the components. The dashed edges are typically called \textit{path-parent} pointers. \url{http://neerc.ifmo.ru/wiki/index.php?title=Link-Cut_Tree}
}
\end{figure}

An important operation of the Link/Cut tree is the \textit{expose(u)} operation. It transforms the virtual tree in a way that puts node $u$ at the root of the virtual tree, and after the operation the path from $u$ to the root of the real tree consists of all the nodes to the right of $u$ in its Splay tree. \textit{expose(u)} works by doing a sequence of splay and relinking operations. Figure~\ref{fig:lca_expose} shows how the tree is transformed with solid edges on the path from $u$ to the root of the real tree.

\begin{figure}
	\includegraphics[width=\textwidth]{../figures/linkcut-expose.png}
	\caption{\label{fig:lca_expose}The real tree before and after exposing node $u$. \url{http://neerc.ifmo.ru/wiki/index.php?title=Link-Cut_Tree}}
\end{figure}

Let us now look at how to answer nearest common ancestor queries. It actually becomes really simple with the use of the \textit{expose} operation. To answer nearest common ancestor of $u$ and $v$ we first expose $u$ to build a path from $u$ to the root of the real tree. Next we expose $v$. Exposing $v$ will create a path from $v$ to the root of the real tree. This path will cross the path just exposed by $u$ in the nearest common ancestor of $u$ and $v$. The expose operation on $v$ will relink the tree such that $u$ hangs off this path in a Splay tree, but not necessarily as the root of this tree. To resolve this we splay $u$ to the root of its tree. Now we can return the \textit{path-parent} of $u$, which must be the nearest common ancestor of $u$ and $v$ by construction of the Link/Cut tree and the expose operation.

It is proved by Sleator and Tarjan~\cite{Sleator1985} that all operations on the tree can be performed with an upper bound of $\mathcal{O}(\log n)$.

\chapter{In-place merging}
\label{chp:inplace}
\subsection{Problem description}
\textit{Describe an algorithm that given an array containing $m+n$ elements $x_1x_2\dots x_m y_1y_2\dots y_n$, where $x_1\leq x_2\leq \dots \leq x_m$ and $y_1\leq y_2\leq \dots \leq y_n$, inplace merges the two sorted sequences. State the running time of your algorithm.}

\textit{What is the best running time you can achieve?}

\subsection{Introduction}
Merging is the process of combining two pre-sorted lists to create a single sorted list. The natural way to merge two lists is to repeatedly compare the smallest elements from both lists and then output the smallest of the two. This is repeated until both lists are exhausted. This process requires $m+n-1$ comparisons and $m+n$ data moves. A merge is called stable if it preserves the order of the elements in the respective lists.

In this exercise we will look at merging two sorted lists in-place. A merge algorithm is said to be in-place if it uses no more than $\mathcal{O}(1)$ extra memory.

A solution to the problem of in-place merging two pre-sorted lists will be presented in the next section. This solution is not stable but is shown to be optimal up to the values of constants of proportionality in the space-time bounds. The solution was first presented by Kronrod in 1969\cite{Kronrod}.

\subsection{Kronrod}
Initially we chop up the two pre-sorted lists into blocks of size $k$, i.e. $k$ is the number of elements in each block. We set $k =\sqrt{m+n}$. The last element in each block is selected. We call this element the mark of each block. The blocks are then rearranged in sorted order according to the mark of each block into a list $L$ of length $m+n$. We can now conclude that each element is at most $k$ positions away from their final position in the merged array.

We now use the last two blocks in $L$ as a temporary output location for our merge operation. This means that the last two blocks become unsorted. When the temporary buffer becomes full, i.e. we have merged two blocks fully, we swap the two merged blocks that reside in the temporary buffer with the old unmerged blocks such that the merged blocks find their correct position in the list. We do this for all blocks until we reach the final two blocks. These blocks are the unsorted temporary buffer. The two buffers can be sorted using selection sort in time $\mathcal{O}(2\cdot\left(\sqrt{m+n}\right)^2) = \mathcal{O}(m+n)$.

\subsection{Time complexity}
The initial rearrangement is a sorting of the marks. We have $\mathcal{O}(\sqrt{m+n})$ such marks and running an insertion sort on these elements will run in $\mathcal{O}( (\sqrt{m+n})^2) = \mathcal{O}(m+n)$ time.

Merging two blocks consumes one comparison per element resulting in $m+n - 2\sqrt{m+n} = \mathcal{O}(m+n)$ comparisons.

Finally the last two blocks which makes up the temporary buffer are sorted using selection sort in time $\mathcal{O}(2\cdot\left(\sqrt{m+n}\right)^2) = \mathcal{O}(m+n)$.

In total we do $3\cdot\mathcal{O}(m+n) = \mathcal{O}(m+n)$ comparisons giving a linear time algorithm in the size of the input which is asymptotically optimal.

%We can also argue about the number of data moves. Block rearrangement consumes 

\chapter{Prefix counting}
\label{chp:prefix}
\subsection{Problem description}
\textit{Let $A$ be an array of size $n$, where each $A\left[i\right]$ is either $0$ or $1$. Describe a data structure that supports updating an entry of $A$ to either $0$ or $1$, and the query $sum_2\left(i\right)$ that returns $\left(A\left[1\right]+A\left[2\right]+\dots+A\left[i\right]\right)\mod 2$, i.e. decides if the sum of the first $i$ entries of $A$ is odd or even.}

\textit{What is the best query time and update time you can achieve? A lower bound of $\Omega(\nicefrac{\log n}{\log\log n})$ is known for the operations on the RAM.}

%\textit{Optional: Generalize the construction to return the sum $A[1]+A[2]+\dots+A[i]$, i.e. the number of entries equal one among the first $i$ entries.}

\subsection{Solution}
We construct a tree of height $\Theta(\nicefrac{\log n}{\log\log n})$ by having a fanout of size $\log n$. The leafs of the tree is the array elements, and internal nodes store the parity of the subtrees. Furthermore we build a table of size $2^{\log n} = n$ storing every combination of parity for blocks of length $\log n$. This table will allow us to calculate the parity of a block in constant time by a single table lookup.

We are now ready to describe how to update and query the structure:
\begin{itemize}
	\item{$Update(i, x)$ - Traverse down the tree to leaf $i$ and update the value. If the value changes then recursively change the parity of all the internal nodes up the tree.}
	\item{$Query(i)$ - Figure~\ref{fig:pre_query} shows a typical query. To answer this query we combine the answers of recursively doing table lookups on the left subtree of the parent of $i$ up to the root. In the example we answer the subquery up to $k$ by a table lookup, which is combined with a table lookup for the missing partial block.}
\end{itemize}

\begin{figure}
\includegraphics[width=\textwidth]{../figures/pre_query.png}
\caption{\label{fig:pre_query}A typical prefix query}
\end{figure}

\subsection{Complexity}
Update will simply traverse the tree up and down and update a single value in each step. This will give a complexity equal to the height of the tree, i.e. a complexity of $\mathcal{O}(\nicefrac{\log n}{\log\log n})$.

A query is answered by recursively doing constant time lookups from the leaf up to the root. Once again this gives a complexity equal to the height of the tree, $\mathcal{O}(\nicefrac{\log n}{\log\log n})$.

These bounds are optimal!

\pagebreak

\bibliography{references}

\end{document}


